\section{Configuración}
La configuración del chat la manejaremos con una clase llamada \textit{Config}.
En la clase tendremos los directorios con ruta absoluta de las diferentes carpetas necesarias.
Habrá variables para controlar cuántos usuarios y conversaciones tenemos registrados en el sistema, y los puertos usados para las comunicaciones.
También tendremos guardadas las extensiones con las que guardaremos los ficheros de usuarios y conversaciones.

Al inicializar una instancia de la clase \textbf{Config} se crearán las carpetas necesarias para el manejo de usuarios y conversaciones.
Tendremos funciones para guardar la configuración actual y para acceder al fichero de cada conversación o usuario. También podremos comprobar si existe un usuario o conversación.